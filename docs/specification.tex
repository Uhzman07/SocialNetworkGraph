\documentclass[12pt,pdftex]{article}

\usepackage{fullpage}
\usepackage{graphicx}
\usepackage{fancyhdr}
\usepackage{url}

\begin{document}
\newcommand{\Term}{Winter 2018}
\newcommand{\Course}{CSCI2110}
\newcommand{\Assignment}{Friend Recommendation}
\newcommand{\Due}{12:00 noon, Monday, March 26, 2018}

\renewcommand{\headheight}{15pt}
\renewcommand{\headsep}{5pt}
\pagestyle{fancy}
\lhead{\bf \Course}
\chead{\bf \Term}
\rhead{\bf \Assignment}
\lfoot{}
\cfoot{\thepage}
\rfoot{}

\title{\vspace*{-12ex}
       \Course\\
       \Assignment}
%\author{Instructor: Alex Brodsky}
%\date{Due: \Due}
\maketitle

\newcommand{\ignore}[1]{}
\newcommand{\noignore}[1]{#1}
\newcommand{\docommand}[1]{\begin{center} \fbox{\bf\tt #1} \end{center}}
\newcommand{\NoItemSpace}{\setlength{\parskip}{0pt}
                          \setlength{\partopsep}{0pt}
                          \setlength{\parsep}{0pt}
                          \setlength{\itemsep}{0pt}}


\newcommand{\Tag}[1]{``{\tt #1}''}

\textbf{Note: This specification is provided as part of Assignment 2, 3 and 4 of CSCI2134 in Winter 2024 and explains what the code is supposed to do. Please see the assignment instructions for your task, e.g. writing unit tests for the code. Do not spend time writing code for the friend recommendation problem.}
\vspace{1em}

The purpose of this assignment is to get you to create and use maps and graphs,
and further improve your programming skills.
% practice a bit more recursion to 

As discussed in class and the first tutorial, for each problem you will 
be provided with a description of the problem and a JUnit test class 
to test your code.  A similar JUnit test class will be used to evaluate
your code.  You can use Eclipse or other IDEs to run the JUnit test class
to test your code.

Your code {\bf must} compile.  If it does not compile, you will receive a 
$0$ on the assignment.


\noignore{
\begin{figure}[h]
\begin{center}
\ \includegraphics[scale=0.50]{figures/social-network-grid.jpg}\ 
\end{center}
\caption{\footnotesize\protect\url{www.stockphotosecrets.com/free-photos/free-vectors/social-network-grid.html}}
\end{figure}
}

\section*{Background: Social Networks}
Social networks are networks that represent relationships between
people.  One common relationship is ``friend''.  This relationship
is bidirectional, e.g., if Alice is friends with Bob, then Bob is
friends with Alice.  Social networks are constantly changing as
users join or leave the network, and friend or unfriend other users.
Many social networks have a ``friend recommender'' feature that
recommends new friends to members if they share a mutual friend.
For example, if Alice is friends with Bob, and Carol befriends Bob,
the system will recommend Alice and Carol to each other. 


\section*{Problem: Recommend Friends}
Given a series of user actions in the social network that is initially empty, 
recommend new possible friendships after one user friends another user.
For example, if Alice, Bob, and Carol are all network members, and Alice
and Bob are friends.  If Bob and Carol become friends, then your program
should recommend that Alice and Carol should become friends.

For example, given the user actions 
\begin{figure}[h]
\begin{quote}
\begin{verbatim}
Alice joins
Carol joins
Bob joins
Bob friends Alice
Bob friends Carol
Dave joins
Dave friends Bob
end
\end{verbatim}
\end{quote}
\caption{Sample user actions.  \label{fig:input}}
\end{figure}
the resulting recommendations would be:
\begin{figure}[h]
\begin{quote}
\begin{verbatim}
Alice and Carol should be friends
Alice and Dave should be friends
Carol and Dave should be friends
\end{verbatim}
\end{quote}
\caption{Friend recommendations based on user actions in 
         Figure~\ref{fig:input}
         \label{fig:output}}
\end{figure}

Your task is to create a program that reads in the user actions and
generates the friend recommendations.

Write a program called {\tt Friend.java} that reads in the user actions
from the console ({\it System.in}) and outputs the corresponding 
friend recommendations.  {\bf Your {\tt Friend} class must implement
the provided {\tt Tester} interface and also contain the {\tt main()}
method where your program starts running.}  This is because your
program will be tested via this interface.  The interface contains
a single method:
\begin{quote}
\begin{verbatim}
public ArrayList<String> compute( Scanner input );
\end{verbatim}
\end{quote}
This method must perform the required computation.

\subsection*{Input}
The {\tt compute()} method takes a {\it Scanner} object, which contains 
one or more lines.  Each line, except the last one, contains a user
action.  There are four user actions:
\begin{quote}
\begin{description}\NoItemSpace
\item[{\tt A joins}]: means that new user $A$ has joins the network.
\item[{\tt A friends B}]: means that user $A$ and $B$ are now friends.
\item[{\tt A unfriends B}]: means that user $A$ and $B$ are no longer friends.
\item[{\tt A leaves}]: means that user $A$ has left the network (deleted 
                     their account).  
%\item[{\tt end}]: end program
\end{description}
\end{quote}
where {\it A} and {\it B} are names of users.  The user names are
single words with no white spaces within them.  The last line of
input is simply the word ``{\tt end}''.  See Figure~\ref{fig:input}
for an example.

Hint:  Use the {\tt nextLine()} method of the {\it Scanner} object to 
read one line of input at a time, and then parse the line.  See the 
solution for Assignment 1 for how to parse input one line at a time.


\subsection*{Semantics}
The social network is initially empty.  All user actions are on the
same network.  

Only users that are not part of the social network can join.  I.e.,
each user will have a unique name.  Users can leave the network
and then rejoin.

When a user leaves the network, the user and all his/her friendships
are deleted from the network.  If a user rejoins the network, it's
as if he/she is joining for the first time.

All friendships are symmetrical.  I.e., ``{\tt A friends B}'' is
the same as ``{\tt B friends A}''.  {\bf No recommendation should
be generated for users who are already friends.}

\subsection*{Output}
The method \verb%compute( Scanner input )% should return an {\it
ArrayList} of {\it String}s denoting the friend recommendations in
the same order as the user actions that caused them.  A recommendation
is of the form
\begin{quote}
{\tt {\it A} and {\it B} should be friends}
\end{quote}
where $A$ and $B$ are user names and {\bf must} be in alphabetical
order.  For example, this recommendation is correct: ``{\tt Alice
and Bob should be friends}'', while this one is not: ``{\tt Bob and
Alice should be friends}''.  If more than one recommendation is
generated by a single user action (as in Figures~\ref{fig:input}
and~\ref{fig:output}), the recommendations should be ordered
alphabetically.

\subsection*{Example}
% Below are a sequence of examples and the corresponding output.

\begin{tabular}{l|l}
{\bf Sample Input } & {\bf Sample Output} \\
\hline
\begin{minipage}[t]{0.3\linewidth}
\vspace*{0.5ex}
\begin{verbatim}
Alice joins
Carol joins
Bob joins
Bob friends Alice
Bob friends Carol
Dave joins
Dave friends Bob
end
\end{verbatim}
\vspace*{0.5ex}
\end{minipage} &
\begin{minipage}[t]{0.5\linewidth}
\vspace*{0.5ex}
\begin{verbatim}
Alice and Carol should be friends
Alice and Dave should be friends
Carol and Dave should be friends
\end{verbatim}
\end{minipage} \\
\end{tabular}

\section*{Hints and Suggestions}
\begin{itemize}\NoItemSpace
\item Use a 1-pass algorithm:  Simply check for 
      friendships after processing each user action.
\item There is not a lot of code to write.  The sample solution is under 
      115 lines of code.
\item Your code {\bf must} compile.  If it does not compile, you will
      receive a $0$ on the assignment.
\item Your code must be well commented and indented.  Please see the
      Assignments section for this course on Brightspace for Code 
      Style Guidelines.
\item You may assume that all input will be correct.  
\item Be sure to test your programs using the provided JUnit test class. 
\end{itemize}


\section*{Grading}
The assignment will be graded based on three criteria: 
\begin{description}\NoItemSpace
\item[Functionality] ``Does it work according to specifications?''.  
   This is determined in an automated fashion by running your program on 
   a number of inputs and ensuring that the outputs match the expected 
   outputs.  The score is determined based on the number of tests that
   your program passes.  So, if your program passes $\frac{t}{T}$ tests,
   you will receive that proportion of the marks.

\item[Quality of Solution] ``Is it a good solution?''  This considers
   whether the solution is correct, efficient, covers boundary conditions,
   does not have any obvious bugs, etc.  This is determined by visual
   inspection of the code.  Initially full marks are given to each solution
   and marks are deducted based on faults found in the solution.

\item[Code Clarity] ``Is it well written?''  This considers whether the
   solution is properly formatted, well documented, and follows 
   coding style guidelines.  
\end{description}
If your program does not compile, it is considered non-functional and of
extremely poor quality, meaning you will receive $0$ for the solution.

\begin{table}[h]
\begin{center}
\begin{tabular}{|l|c|} \hline
                                   & {\bf Marks} \\ \hline
Functionality                      &   20   \\
Quality of Solution                &   20   \\ 

{\bf Code Clarity}                 &   10   \\ \hline
\rule{0mm}{5mm} {\large\bf Total}   &  50  \\ \hline
\end{tabular}
\end{center}
\caption{Marking scheme for \Assignment{}.}
\end{table}

\vspace*{-2ex}
\section*{What to Hand In}
\vspace*{-1ex}
Submit the source files for your program in a zip file (\verb%.zip%).
You should have at least one source file: {\tt Friend.java}.
If you are using Eclipse, we recommend that you submit the entire
project bundle.  Submission is to be done via Brightspace.  Your
code {\bf must} compile.  If it does not compile, you will receive
a $0$ on the assignment.

\ignore{
\newpage
\thispagestyle{empty}

\begin{center}
\LARGE{\bf \Course:  \Assignment}\\
\normalsize \Term
\end{center}

\begin{center}
\LARGE
\begin{tabular}{|l|l|} \hline
Student Name   & Student Number \\ \hline
               &                \\ \hline
\end{tabular}
\end{center}

\begin{center}
\Large
\begin{tabular}{|l|c|} \hline
                                   & {\bf Mark} \\ \hline
{\bf Hello World} (3 pts)         &         \\
Functionality                      &    \hfill / 1  \\
Quality of Solution                &    \hfill / 2  \\ \hline

{\bf Prime Factorization} (7 pts)  &         \\
Functionality                      &    \hfill / 3    \\
Quality of Solution                &    \hfill / 4  \\ \hline

{\bf Aggregates} (7 pts)           &         \\
Functionality                      &   \hfill / 3   \\
Quality of Solution                &   \hfill / 4   \\ \hline

{\bf Code Clarity}                 &   \hfill / 3   \\ \hline
\rule{0mm}{8mm} {\huge\bf Total}     &   \hfill / 20  \\ \hline
\end{tabular}
\end{center}

\noindent {\LARGE \bf Comments:}
}

\ignore{
\section{Grading}
Each summary is worth 20\% of the assignment mark.  The bonus portion
is worth an additional 10\% of the assignment mark.  Each of the
five application summaries and the bonus part will be graded according
to the following rubric:\footnote{Based in part on Fleming, ``Grading
Rubric for Written Assignments'', CSCI 2100, 2011}



\hyphenpenalty=0
\newcommand{\desc}[1]{\parbox[t]{1.18in}{\sloppy \footnotesize #1\vspace*{1ex}}}
%\newcommand{\desc}[1]{\begin{minipage}[t]{1.16in}\sloppy \raggedright \small #1\vspace*{1ex}\end{minipage}}
\begin{table}[h]
\begin{center}
\sloppy
\begin{tabular}{|l|l|l|l|l|}
\hline
                    & Exceptional: A & Acceptable: B & Substandard: C-D & Unacceptable: F \\
\hline
\hline

\parbox[t]{0.90in}{\bf Novelty \\ (20\%)}
                    & \desc{Application is exceptionally novel and uses
                       mobile technologies in new and interesting ways.}
                    & \desc{Application is somewhat novel and uses
                       mobile technologies in interesting combinations.}
                    & \desc{Application is mundane but uses
                       mobile technologies in useful combinations.}
                    & \desc{Application is mundane novel and uses
                       mobile technologies in mundane ways.} \\
\hline
\parbox[t]{0.90in}{\bf Content \\ (30\%)}
                    & \desc{Ideas well organized and logically laid out always 
                       or almost always.  Shows~superior understanding of the 
                       subject. } 
                    & \desc{Ideas well organized and logically laid out with 
                       competence. Shows 
                       commonplace understanding of the subject. } 
                    & \desc{Minimal organization and logical progression of 
                       ideas. Shows partial 
                       familiarity with the subject.} 
                    & \desc{Little or no organization or logical flow of ideas. 
                       Shows a great deal of 
                       misunderstanding about the subject. } \\
\hline
\parbox[t]{0.90in}{\bf Analysis \\ (30\%)}
                    & \desc{Addresses all points. Justifies all design 
                       decisions.  Discusses criteria used in justifications.   
                       Considers the issues from multiple points of view.}
                    & \desc{Addresses most of the points.  Justifies most 
                       design decisions.  Identifies criteria used in 
                       justifications.} 
                    & \desc{Addresses some of the points. Justifies some design
                       decisions.  Identifies some criteria in the analysis.}
                    & \desc{Addresses few of the points.  Little or no 
                       justification and no identification of the criteria 
                       used.}\\
\hline
\parbox[t]{0.90in}{{\bf Writing \\ (20\%)\\} \footnotesize\sloppy
                    Style and tone, standard conventions for grammar, word use, 
                    spelling, citations, headings, paragraphs, figures, and 
                    tables. \vspace*{1ex}}
                    & \desc{Shows exceptional use of tone and style. Speaks 
                      to the reader with precise, concise, appropriate 
                      language and choice of words.  Always uses standard 
                      conventions.  The document looks professional.}
                    & \desc{Shows competent use of tone and style. Makes good 
                      word choices.  Mostly uses standard conventions.  The 
                      document could use some editing.}
                    & \desc{Shows minimal attention to tone and style. 
                      Shows poor usage or ineffective word variation.
                      Does not consistently use standard conventions.  
                      The document requires significant editing.}
                    & \desc{Shows little or no understanding of appropriate 
                      tone. Uses inappropriate language and word choice.
                      Standard conventions are flouted.  Document is 
                      unreadable.} \\
\hline
\end{tabular}
\end{center}
\end{table}

}
\end{document}
